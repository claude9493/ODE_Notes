\documentclass{article}
\usepackage{amsmath}
\usepackage{amsfonts}
\setlength\parindent{0em}

\title{First order linear ODEs}
\author{Yun Zhang}
\date{}
\begin{document}
\maketitle
\section{Method of Integrating Factors}
\subsection{Restriction}
This method can be applied to \textbf{$1_{st}$ order linear equation}, which has such standard form:
\begin{equation}
	\frac{dy}{dt}+p(t)y = g(t) 
\end{equation}
% \[\frac{dy}{dt}+p(t)y = g(t) \]
\subsection{Description}
\begin{itemize}
	\item First, find the \textbf{integrating factor} function $\mu(t)$ by:
	\begin{equation}
			 \mu(t) = e^{\int{p(t)dt}} 
	\end{equation}
	\item Then, we can get the solution by: 
	\begin{equation}
		y = \frac{1}{\mu(t)}(\int \mu(t)g(t)dt+c)\text{ , c is a constant}
	\end{equation}
%	\[y = \frac{1}{\mu(t)}(\int \mu(t)g(t)dt+c)\text{ , c is a constant} \]
	\item Check whether there is any equilibrium solution. And if there is any initial condition, use it to get the constant $c$.
\end{itemize}

\section{Method of constant variation}
\subsection{Restriction}
This method has same restriction with the method of integrating factors.
\subsection{Description}
\begin{itemize}
	\item First, change euqation (1) to a simpler ODE by let $g(t)$ to be zero. The original equation is changed to 
	\begin{equation}
			\frac{dy}{dt}+p(t)y = 0
	\end{equation}
	which is called to be \textbf{homogeneous}.
	\item Then, solve the new ODE, and we can get its solution $y = ce^{-\int p(t)dt}$
	\item Let $c$ to be a function of $t$ (i.e. $c(t)$), and substititute it back to the original equation:
	\[c'(t)e^{-\int p(t)dt} + c(t)(e^{-\int p(t)dt})' + p(t)c(t)e^{-\int p(t)dt} = g(t)\]
	and we can get:
	\begin{equation}
		c'(t)e^{-\int p(t)dt} = g(t)
	\end{equation}
	\item From the above equation, we can easily solve $c(t)$ with a new constant$C$, then we can get $y(t)$ by substuting $c(t)$ to the relation of $y(t)$ and $c(t)$
	\item Check whether there is any equilibrium solution. And if there is any initial condition, use it to get the constant $C$.
\end{itemize}

\section{Separable Equation}
\subsection{Restriction}
\textbf{Separable} equations are ODEs which are in the form:
\begin{equation}
	M(x)+N(y)\frac{dy}{dx} = 0
\end{equation}
, where $M$ is a fuction of $x$ only, and $N$ is a function of $y$ only. Because if it is written in the differentional form:
\begin{equation}
	M(x)dx+N(x)dy = 0
\end{equation}
terms can be placed on opposite sides of the equation, which looks more symmetric.
\subsection{Description}
To solve a separable equation, we return to Eq.(6), and let $H_1(x)$ and $H_2(x)$ to be any antiderivatives of $M$ and $N$, respectively. And Eq.(6) becomes
\begin{equation}
	H_1'(x) +H_2'(y)\frac{dy}{dx} = 0
\end{equation}
If $y$ is regarded as a function of $x$, then according to the chaing rule,
\[H_2'(y)\frac{dy}{dx} = \frac{d}{dy}H_2(y)\frac{dy}{dx} = \frac{d}{dx}H_2(y) \]
Then we can write Eq.(8) as:
\begin{equation}
	\frac{d}{dx}\left[ H_1(x)+H_2(x) \right] = 0
\end{equation}
By integrating it, we obtain:
\begin{equation}
	H_1(x)+H_2(y) = c
\end{equation}
where c is an arbitrary canstant. Eq.(10) defines the solytion \textbf{implicitly}. If it's obvious to see the explicit solution, you can change Eq.(10) to a explicit form.\\

Finally, check whether there is any equilibrium solution. And if there is any initial condition, use it to get the constant $c$.

\section{Exact Equation}
\subsection{Restriction}
Let the differeatial equation
\begin{equation}
	M(x,y)+N(x,y)y' = 0
\end{equation}
be given. Suppose we can identify a function $\psi(x,y)$ s.t.
\begin{equation}
	\frac{\partial\psi}{\partial x}(x,y) = M(x,y),\qquad\frac{\partial\psi}{\partial y}(x,y) = N(x,y)
\end{equation}
and such that $\psi(x,y)=c$ defines $y=\phi(x)$ implicitly as a differantiable function of $x$. Then the differential equation Eq.(11) becomes
\begin{equation}
	\frac{d}{dx}\psi\left[x, \phi(x)\right] = 0
\end{equation}
In this case, Eq.(11) is said to be an \textbf{exact} differential equation.
\subsection{Description}
First we check whether a given differential equation is exact by computing $M_y$ and $N_x$ from Eqs.(12) and check whether they are equal. If so, the equation is exact. 

Then, we integrate the first of Eqs.(12) with respect to $x$ and obtain:
\begin{equation}
	\psi(x,y) = Q(x,y)+h(y)
\end{equation}
The function $h$ in Eq.(14) is an arbitrary differential funciton of $y$. Then we combine Eq.(14) with the second of Eq.(12), we can get:
\[\psi_y(x,y) = \frac{\partial Q}{\partial y}(x,y) + h'(y) = N(x,y) \]
Solving it for $h'(y)$, we have
\begin{equation}
	h'(y) = N(x,y) - \frac{\partial Q}{\partial y}(x,y)
\end{equation}
It is easy to show that the right side of Eq.(16) deos not depend on $x$. Then we can find $h(y)$ by integrating Eq.(16). Substitute it back to Eq.(14), we can get the implicit solution.

Finally, do not forget to seek the equilibrium solution and computing the constant if there is any initial condition.
\subsection{Strict definition of exact equation}
Let the function $M$, $N$, $M_y$, and $N_x$ be \textbf{continuous} in the rectangular region $R$:  $\alpha < x <\beta,\quad \gamma < y <\delta$. Then Eq.(11)\[M(x,y)+N(x,y)y' = 0\] is an exact differential equation in $R$ iff
\begin{equation}
	M_y(x,y) = N_x(x,y)
\end{equation}
at each point of R. That is there exists a funtion $\psi$ satisfying Eqs.(12) iff $M$ and $N$ satisfy Eq.(16).
\end{document}