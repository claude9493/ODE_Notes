\documentclass{article}
\usepackage{amsmath}
\usepackage{amsfonts}
\setlength\parindent{0em}

\title{Second order linear Equations}
\author{Yun Zhang}
\date{}
\begin{document}
\maketitle

\section{Make the problem clear}
A second order ordinary differential equation has the form
\begin{equation}
	\frac{d^2y}{dt^2} = f\left(t, y, \frac{dy}{dt}\right)
\end{equation}
It is said to be \textbf{linear} if the function $f$ has the form
\begin{equation}
	f\left(t,y,\frac{dy}{dt}\right) = g(t) - p(t)\frac{dy}{dt} - q(t)y 
\end{equation}
here $g$, $p$, $q$ are specified funcitons of $t$, the independent variable. We urally rewirite the Eq.(1) as 
\begin{equation}
	y'' + p(t)y' + q(t) y = g(t)
\end{equation}
Also we often see the equation
\begin{equation}
	P(t)y'' + Q(t)y' +R(t)y = G(t)
\end{equation}
If $P(t) \neq 0$ we can divided Eq.(4) by $P(t)$ to obtain Eq.(3).

\subsection*{homogeneous and nonhomogeneous}

$\qquad$A second order linear equation is said to be \textbf{homogeneous} if the term $g(t)$ in Eq.(3) or the term $G(t)$ in Eq.(4) is zero for all $t$. Otherwise the equation is called \textbf{nonhomogeneous}. The term $g(t)$ or $G(t)$ is sometimes called the nonhomogeneous term.


\section{Homogeneous Equatioins with Constant Coefficients}
The equation we discussed in this section has the form
\begin{equation}
	ay''+by'+cy = 0
\end{equation}
where $a,b,c$ are give constants.
\subsection{characteristic equation}
We start by seeking exponential solutions of the frm $y = e^{rt}$, $r$ is a parameter to be determined. Then it follows that $y' = re^{rt}$ and $y'' = r^2e^{rt}$. Substitute them for $y$, $y'$ and $y''$ in Eq.(5), we obtain
\[(ar^2+br+c)e^{rt} = 0\]
or, since $e^{rt}\neq 0$,
\begin{equation}
	ar^2+br+c = 0
\end{equation}
Eq.(6) is called the characteristic equation for the differential equation (5).\\
Then we begin to discuss the root of Eq.(6).
\subsection{$b^2 - 4ac > 0$ two different real roots}
Suppose the two different roots of the characteristic equation is $r_1$ and $r_2$. Then $y_1(t) = e^{r_1t}$ and $y_2(t) = 3^{r_2t}$ are two solutions of Eq.(5). Moreover, the general solution can be written as 
\begin{equation}
	y = c_1y_1(t) + c_2y_2(t) = c_1e^{r_1t}+c_2e^{r_2t}
\end{equation}
Since the general solutioni is in such form, the solution has a relatively simple geometrical behavior: 
\begin{quote}
	As $t$ increases, the magnitude of the solution either tends to zero (both exponents are negative) or else grows rapidly (at least one exponent is positive).
\end{quote}
If we have the initial conditions that \[y(t_0) = y_0\text{,  } y'(t_0) = y'_0\]
By substituting $t=t_0$ and $y = y_0$ in Eq.(7) we obtain
\begin{equation}
	c_1e^{r_1t_0}+c_2e^{r_2t_0} = y_0
\end{equation}
and
\begin{equation}
	c_1r_1e^{r_1t_0}+c_2r_2e^{r_2t_0} = y'_0
\end{equation}
So solving the two equations, we can find that:
\begin{equation}
	c_1 = \frac{y'_0-y_0r_2}{r_1-r_2}e^{-r_1t_0} \text{,  } c_2 = \frac{y_0r_1-y'_0}{r_1-r_2}e^{-r_2t_0}
\end{equation}
Recall that $r_1-r_2 \neq 0$, so Eq.(10) always make sense.


\subsection{$b^2 - 4ac < 0$ complex and conjugate roots}
Suppose now that $b^2-4ac$ is negative. Then the roots of Eq.(6) are conjugate complex numbers.
We denote them by 
\begin{equation}
	r_1 = \lambda + i\mu\text{,  } r_2 = \lambda -i\mu
\end{equation}
Where $\lambda\text{ and }\mu$ are real. Then the corresponding expressions for $y$ are
\begin{equation}
	y_1(t) = exp\left[(\lambda+i\mu)t\right]\text{,  } 	y_2(t) = exp\left[(\lambda-i\mu)t\right]
\end{equation}
\subsubsection*{Euler's Formula}
From Taylor's series, we have
\begin{align*}
e^{it} &= \sum_{n=0}^{\infty}\frac{(it)^n}{n!}\\
	   &= \sum_{n=0}^{\infty}\frac{(-1)^nt^{2n}}{(2n)!} + i\sum_{n=1}^{\infty}\frac{(-1)^{n-1}t^{2n-1}}{(2n-1)!}\\
	   &= cost+i sint
\end{align*}
So 
\begin{equation}
	e^{(\lambda+i\mu)t} = e^{\lambda t}cos\mu t+ie^{\lambda t}sin\mu t
\end{equation}
Chosing the real part of of either $y_1(t)$ and $y_2(t)$, we obtain the general solution of Eq.(5)
\begin{equation}
	y = c_1e^{\lambda t}cos\mu t + c_2e^{\lambda t}sin\mu t
\end{equation}

\subsection{$b^2 - 4ac = 0$ repeated root}


\end{document}